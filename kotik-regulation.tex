\documentclass[a4paper,11pt]{article}

\usepackage[utf8]{inputenc}
\usepackage[T1]{fontenc}
\usepackage[polish]{babel}
\usepackage[MeX]{polski}
\usepackage{listings}
\selectlanguage{polish}
\usepackage{graphicx}
\usepackage{setspace}

\renewcommand{\paragraph}[1] {\begin{center}§ {#1}\end{center}}
\newcommand{\content}[1] {\begin{flushleft}{#1}\end{flushleft}}
\newcommand{\chapter}[2] {\begin{center}\section*{ROZDZIAŁ {#1} \\ {#2}}\end{center}}

\begin{document}

\begin{titlepage}
	\begin{center}
		\vspace{20pt}
			{\LARGE Regulamin} \\
			\vspace{10pt}
				\begin{huge}
					\textsc{\textbf{Koła Otwartych}} \\
					\vspace{5pt}
					\textsc{\textbf{Technik Informacyjnych}} \\
					\vspace{14pt}
					\textsc{\textbf{i Komputerowych}}
				\end{huge}
		\vspace{40pt}
		
		\includegraphics[scale=0.4]{duzy-kotik.png}
		
		\vfill
		
		\textit{przy Wydziale Elektrycznym} \\
		\textit{Politechniki Warszawskiej} \\
		\vspace{10pt}
		\includegraphics[scale=0.3]{ee.jpg}
			
		\vfill
		
		\today
	\end{center}
\end{titlepage}

\chapter{I}{Postanowienia ogólne}

\paragraph{1}
\content{Nazwa Koła brzmi Koło Otwartych Technik Informacyjnych \\  i Komputerowych, zwane dalej Koło działa przy Wydziale Elektrycznym Politechniki Warszawskiej.}

\paragraph{2}
\content{Koło działa zgodnie z obowiązującym prawem w szczególności \\ z ustawą z dnia 27 lipca 2005 r.}

\paragraph{3}
\content{Prawo  o  szkolnictwie  wyższym (Dz.  U.  nr  164,  poz.  1365  z  późn.  zm.)  oraz  ze  Statutem Politechniki Warszawskiej.}

\paragraph{4}
\content{Koło nie posiada osobowości prawnej i działa w ramach struktury Politechniki Warszawskiej.}

\chapter{II}{Cele Koła}

\paragraph{5}
\content{Celem Koła jest ułatwienie jego członkom rozwijania i pogłębiania wiedzy w zakresie szeroko pojętej  informatyki,  a  w  szczególności  otwartych  standardów,  oprogramowania  i  rozwiązań technicznych.}

\paragraph{6}
Koło realizuje cele, o których mowa w § 5, przez: 
\begin{enumerate}
	\item organizowanie konferencji, seminariów oraz odczytów związanych z celami Koła;
	\item rozpatrywanie zagadnień naukowo-badawczych;
	\item inspirowanie tematyki ćwiczeń, prac projektowych oraz prac dyplomowych;
	\item współpracę z innymi kołami, stowarzyszeniami i ośrodkami naukowymi;
	\item propagowanie otwartych standardów w technikach informatycznych;
	\item prowadzenie strony internetowej Koła;
\end{enumerate}

\chapter{III}{Członkowie Koła}

\paragraph{7}
\content{Członkiem  Koła  może  zostać  każdy  student  lub  doktorant  Politechniki  Warszawskiej zainteresowany realizacją celów Koła.}

\paragraph{8}
\content{Członkami honorowymi Koła mogą być absolwenci Politechniki Warszawskiej lub zatrudnieni w
niej nauczyciele akademiccy, a także inne osoby, które położyły szczególne zasługi dla rozwoju Koła.}

\paragraph{9}
\begin{enumerate}
	\item Członkostwo nabywa się na mocy uchwały Zarządu, wydanej po rozpatrzeniu pisemnego wniosku złożonego przez kandydata na członka.
	\item Kandydatowi na członka Koła przysługuje odwołanie od uchwały Zarządu w przedmiocie odmowy przyjęcia na członka do Walnego Zebrania w terminie 7 dni od otrzymania uchwały.
	\item Decyzja Walnego Zebrania, o której mowa w ust. 2, jest ostateczna.
\end{enumerate}

\paragraph{10}
Członek Koła ma prawo:
\begin{enumerate}
	\item uczestniczenia w Walnym Zebraniu;
	\item korzystania z pomocy pracowników naukowych współpracujących \\ z Kołem;
	\item zgłaszania opinii, wniosków i postulatów pod adresem władz Koła;
	\item czynnego i biernego prawa wyborczego;
\end{enumerate}

\paragraph{11}
\content{Członek Koła ma obowiązek:}

\begin{enumerate}
	\item przestrzegać postanowień niniejszego Regulaminu oraz uchwał organów Koła;
	\item aktywnie uczestniczyć w pracach Koła;
	\item dbać o pozytywny wizerunek Koła i Politechniki Warszawskiej;
\end{enumerate}

\paragraph{12}
\content{Członkostwo ustaje na skutek:}
\begin{enumerate}
	\item wykluczenia  –  dokonanego  w  drodze  uchwały  Walnego  Zebrania,  \\ w  przypadku
postępowania  członka  sprzecznego  z  niniejszym  Regulaminem,  Statutem  Politechniki
Warszawskiej  lub w  inny sposób nie  dającego się  pogodzić z  obowiązkami  studenta
Politechniki Warszawskiej;
	\item wykreślenia – dokonanego w drodze uchwały Zarządu, w przypadku utraty przez członka
statusu  studenta  lub  doktoranta  Politechniki  Warszawskiej  oraz  na  wniosek  samego
członka;
	\item śmierci członka.
\end{enumerate}

\chapter{IV}{Władze Koła}

\paragraph{13}
\content{Władzami Koła są:}
\begin{enumerate}
	\item Zarząd;
	\item Walne Zebranie Członków Koła, zwane dalej Walnym Zebraniem.
\end{enumerate}

\paragraph{14}
\content{Do kompetencji Walnego Zebrania należy:}
\begin{enumerate}
	\item określanie kierunków pracy Koła;
	\item wybór, odwoływanie i uzupełnianie składu Zarządu;
	\item udzielanie absolutorium z działalności Zarządu za dany rok akademicki;
	\item uchwalanie zmian w Regulaminie Koła;
	\item podjęcie uchwały w sprawie rozwiązania Koła.
\end{enumerate}

\paragraph{15}
\begin{enumerate}
	\item Zarząd zwołuje Walne Zebranie przynajmniej raz w roku.
	\item Zarząd ma obowiązek zwołać Walne Zebranie także na wniosek co najmniej  25\% członków 
Koła, w terminie nie przekraczającym 14 dni od daty wpłynięcia wniosku.
\end{enumerate}

\paragraph{16}
\content{Uchwały organów Koła podejmowane są zwykłą większością głosów \\ w obecności co najmniej 1/2 uprawnionych do głosowania, o ile niniejszy Regulamin nie stanowi inaczej.}

\paragraph{17}
\begin{enumerate}
	\item Uchwały organów Koła nie mogą  naruszać powagi lub interesu Politechniki Warszawskiej.
	\item Uchwały organów Koła mogą być uchylone przez Rektora Politechniki Warszawskiej w
przypadku ich niezgodności z przepisami prawa, Statutem Politechniki Warszawskiej lub
niniejszym Regulaminem.
\end{enumerate}

\paragraph{18}
\begin{enumerate}
	\item Zarząd jest wybierany przez Walne Zebranie na okres jednego roku akademickiego.
	\item Zarząd składa się co najmniej z 3 członków Koła.
	\item Zarząd wybiera ze swego grona Prezesa, Zastępcę oraz Sekretarza.
\end{enumerate}

\paragraph{19}
\content{Oświadczenia woli w imieniu Zarządu oraz Koła może składać Prezes Koła bądź inna osoba
upoważniona przez Prezesa.}

\paragraph{20}
\content{Zarząd lub członek Zarządu może być odwołany przez Walne Zebranie na wniosek co najmniej
25\% liczby członków Koła.}

\paragraph{21}
\content{Do kompetencji Zarządu należy:}
\begin{enumerate}
	\item reprezentowanie Koła;
	\item wykonywanie uchwał Walnego Zebrania;
	\item przyjmowanie oraz wykreślanie członków Koła;
	\item kierowanie bieżącą działalnością Koła;
	\item podejmowanie  decyzji  we  wszystkich  sprawach  nie  zastrzeżonych  do  kompetencji
Walnego Zebrania.
\end{enumerate}

\chapter{V}{Opiekunowie Koła}

\paragraph{22}
\begin{enumerate}
	\item Opiekunami naukowymi Koła mogą być pracownicy Politechniki Warszawskiej. 
	\item Opiekuni naukowi Koła wybierani są na czas nieokreślony.
\end{enumerate}

\paragraph{23}
\content{Do kompetencji Opiekunów Koła należy:}
\begin{enumerate}
	\item wspieranie i koordynacja działalność Koła;
	\item zatwierdzanie  tematów  badawczych  podejmowanych  przez  Koło  i  merytoryczna
pomoc w ich realizacji.
\end{enumerate}

\chapter{VI}{Majątek i finanse Koła}

\paragraph{24}
\begin{enumerate}
	\item Praca w organach Koła ma charakter społeczny.
	\item Dla realizacji swoich celów Koło korzysta ze środków finansowych:
	\begin{enumerate}
		\item przyznanych przez Uczelnię z jej budżetu;
		\item przekazanych przez sponsorów w formie darowizny;
		\item wypracowanych przez członków koła i przekazanych na konto uczelni na podstawie
	umów zawieranych przez uczelnię z innymi podmiotami;
		\item dobrowolnych  składek  członków  Koła w  przypadku  zatwierdzenia  ich przez  Walne
Zebranie członków.
	\end{enumerate}
\end{enumerate}

\chapter{VII}{Postanowienia końcowe}

\paragraph{25}
\begin{enumerate}
	\item Zmiany  Regulaminu  wprowadza  się  w  drodze  uchwały  Walnego  Zebrania  przyjętej
bezwzględną  większością  głosów  w  obecności  co  najmniej  połowy  uprawnionych  do
głosowania.
	\item Uchwała w sprawie zmiany Regulaminu wymaga zatwierdzenia przez Rektora Politechniki
Warszawskiej.
\end{enumerate}

\paragraph{26}
\content{Rozwiązanie Koła następuje w drodze:}
\begin{enumerate}
	\item uchwały Walnego Zebrania przyjętej bezwzględną większością głosów w obecności co
najmniej połowy uprawnionych do głosowania członków Koła;
	\item uchwały Senatu Politechniki Warszawskiej, podjętej na wniosek Rektora Politechniki
Warszawskiej.
\end{enumerate}
\end{document}
